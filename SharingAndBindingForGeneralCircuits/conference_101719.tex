\documentclass[conference]{IEEEtran}
\IEEEoverridecommandlockouts
% The preceding line is only needed to identify funding in the first footnote. If that is unneeded, please comment it out.
\usepackage{cite}
\usepackage{amsmath,amssymb,amsfonts}
\usepackage{algorithmic}
\usepackage{graphicx}
\usepackage{textcomp}
\usepackage{xcolor}
\def\BibTeX{{\rm B\kern-.05em{\sc i\kern-.025em b}\kern-.08em
    T\kern-.1667em\lower.7ex\hbox{E}\kern-.125emX}}
\begin{document}
\begin{titlepage}

\title{Sharing and Binding for General Circuits\\}

\author{\IEEEauthorblockN{Benedikt Lipinski}
\IEEEauthorblockA{\textit{Interaktionstechnik und Design)} \\
\textit{Hochschule Hamm Lippstadt}\\
Lippstadt, Germany \\
benedikt.lipinski@stud.hshl.de}
}
\maketitle
\begin{abstract}

\end{abstract}
\end{titlepage}
\newpage

\section{Introduction}
\subsection{Problemstellung}
\subsection{Logische Bausteine}
\subsection{Allocation}
\subsection{Binding}
\subsection{Sharing}
\section{Grundlegendes}
\subsection{Kompatiblitäts- und Konfliktgraphen}
\subsection{Strategien zur Architektur Optimierung}
\subsection{Resource Dominated circuits}
\section{General Circuits}
\subsection{Allgm.}
\subsection{Baugruppen}
\section{Sharing and Binding for General Circuits}
\subsection{Unconstrained minimum - Area Binding}
\subsection{Performance Constrained Binding}
\subsection{Performance Directed Binding}
\section{Gleichnis,vorgestellter Algorithmen}
\section{Ausblick}

\begin{thebibliography}{00}
\bibitem{b1}
\end{thebibliography}
\vspace{12pt}


\end{document}
