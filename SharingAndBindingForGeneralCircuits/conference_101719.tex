\documentclass[conference]{IEEEtran}
\IEEEoverridecommandlockouts
% The preceding line is only needed to identify funding in the first footnote. If that is unneeded, please comment it out.
\usepackage{cite}
\usepackage{amsmath,amssymb,amsfonts}
\usepackage{algorithmic}
\usepackage{graphicx}
\usepackage{textcomp}
\usepackage{xcolor}
\def\BibTeX{{\rm B\kern-.05em{\sc i\kern-.025em b}\kern-.08em
    T\kern-.1667em\lower.7ex\hbox{E}\kern-.125emX}}
\begin{document}
\begin{titlepage}

\title{Sharing and Binding for General Circuits\\}

\author{\IEEEauthorblockN{Benedikt Lipinski}
\IEEEauthorblockA{\textit{Interaktionstechnik und Design)} \\
\textit{Hochschule Hamm Lippstadt}\\
Lippstadt, Germany \\
benedikt.lipinski@stud.hshl.de}
}
\maketitle
\begin{abstract}

\end{abstract}
\end{titlepage}
\newpage

\section{Introduction}
Durch das menschliche verlangen Problemstellungen zu-gunsten schwindender Komplexität und erhöhtem Komfort zu automatisieren und zu brechen, wurden genau diese Anforderungen auf die elektrischer Schaltungen umgelegt. 
\subsection{Problemstellung}
Nicht erst durch Moderne Entwicklungen, wie digitale Vernetzung mit dem Internet der dinge  oder das Autonome Fahren wird die Betrachtung,  Zeitliche und Finanzielle Aspekte für die entwicklung multidimensionaler Systeme notwendig. Allerdings lassen genau diese Entwicklungen, die Komplexität sprunghaft ansteigen.

\subsection{Logische Bausteine}
Realisierbar werden elektrische Verschaltungen in der Digitaltechnik erst durch den einsatz sogenannter Logikbausteine, die mit einander kombiniert ein vorher genau bestimmtes verhalten der geplanten Schaltung verursachen.
Hierbei übernehmen verschiedene Bausteine verschiedene Aufgaben, die mal weniger und mal mehr kompliziertere Operationen beinhalten. 
\subsubsection{AND-Gatter}
Eine der Grundoperationen, die sich in fast jeder 
\subsection{Allocation}
\subsection{Binding}
\subsection{Sharing}
\section{Grundlegendes}
\subsection{Kompatiblitäts- und Konfliktgraphen}
\subsection{Strategien zur Architektur Optimierung}
\subsection{Resource Dominated circuits}
\section{General Circuits}
\subsection{Allgm.}
\subsection{Baugruppen}
\section{Sharing and Binding for General Circuits}
\subsection{Unconstrained minimum - Area Binding}
\subsection{Performance Constrained Binding}
\subsection{Performance Directed Binding}
\section{Gleichnis,vorgestellter Algorithmen}
\section{Ausblick}

\begin{thebibliography}{00}
\bibitem{b1} Jürgen Teich, Christian Haubelt, Digitale Hardware/Software-Systeme, Synthese und Optimierung, 2. Auflage,Springer,2007
\end{thebibliography}
\vspace{12pt}


\end{document}
